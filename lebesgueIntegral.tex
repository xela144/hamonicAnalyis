\documentclass[12pt]{article}

% This first part of the file is called the PREAMBLE. It includes
% customizations and command definitions. The preamble is everything
% between \documentclass and \begin{document}.

\usepackage[margin=1in]{geometry}  % set the margins to 1in on all sides
\usepackage{graphicx}              % to include figures
\usepackage{amsmath}               % great math stuff
\usepackage{amsfonts}   
           % for blackboard bold, etc
\usepackage{amsthm}    % better theorem environments

%\usepackage[demo]{graphicx}
%\usepackage{caption}
%\usepackage{subcaption}

% to anchor figures
%\usepackage{float}    
%\restylefloat{figure}

% various theorems, numbered by section

\newtheorem{thm}{Theorem}[section]
\newtheorem{lem}[thm]{Lemma}
\newtheorem{prop}[thm]{Proposition}
\newtheorem{cor}[thm]{Corollary}
\newtheorem{conj}[thm]{Conjecture}

\DeclareMathOperator{\id}{id}

\newcommand{\bd}[1]{\mathbf{#1}}  % for bolding symbols
\newcommand{\mc}[1]{$\mathcal{#1}$}  % for bolding symbols - no math mode
\newcommand{\MC}[1]{\mathcal{#1}}  % for bolding symbols
\newcommand{\RR}{\mathbb{R}}      % for Real numbers
\newcommand{\ZZ}{\mathbb{Z}}      % for Integers
\newcommand{\NN}{\mathbb{N}}      % for Naturals 
\newcommand{\col}[1]{\left[\begin{matrix} #1 \end{matrix} \right]}
\newcommand{\comb}[2]{\binom{#1^2 + #2^2}{#1+#2}}



%%%% "fancy header"
\usepackage{fancyhdr}   %from donez
\pagestyle{fancy}
\headheight 15pt
\rhead{Fall 2015}
\chead{Ch 2. Measure Theory}
\lhead{Real Analysis with Wavelets}
\rfoot{}
\cfoot{\thepage}
\lfoot{}
\renewcommand{\footrulewidth}{0pt}
%\renewcommand{\headrulewidth}{0pt}
%%%%%

%\usepackage{lmodern}
%\renewcommand{\sfdefault}{lmss}
%\usepackage[T1]{fontenc}
%
%
%\makeatletter
%\special{papersize=\the\paperwidth,\the\paperheight}
%
%\usepackage{lipsum}
%\usepackage{marginnote}
%\usepackage[top=1.5cm, bottom=2cm, outer=5cm, inner=2cm, heightrounded, marginparwidth=4.2cm, marginparsep=.5cm]{geometry}
%
%\makeatother
%\usepackage[backend=biber, style=brent]{biblatex}

%\pagestyle{empty}

%%%% cut here

%\usepackage{paralist}
%\renewenvironment{thebibliography}[1]{%
%  \section*{\refname}%
%  \let\par\relax\let\newblock\relax%
%  \inparaenum[{[}1{]}]}{\endinparaenum}
%\renewcommand{\bibitem}[1]{\item}
%\makeatletter
%\renewenvironment{thebibliography}[1]
%     {\section*{\refname}%
%      \@mkboth{\MakeUppercase\refname}{\MakeUppercase\refname}%
%      \begin{inparaenum}%
%      \list{\@biblabel{\@arabic\c@enumiv}}%
%           {\settowidth\labelwidth{\@biblabel{#1}}%
%           \leftmargin\z@  \labelwidth
%           % \leftmargin\labelwidth
%            \advance\leftmargin\labelsep
%            \@openbib@code
%            \usecounter{enumiv}%
%            \let\p@enumiv\@empty
%            \renewcommand\theenumiv{\@arabic\c@enumiv}}%
%      \sloppy
%      \clubpenalty4000
%      \@clubpenalty \clubpenalty
%      \widowpenalty4000%
%      \sfcode`\.\@m}
%     {\def\@noitemerr
%       {\@latex@warning{Empty `thebibliography' environment}}%
%      \endlist\end{inparaenum}}
%\renewcommand\newblock{}
%\makeatother

%%%% cut here 

\usepackage{enumerate}

\theoremstyle{definition}
\newtheorem{definition}{Definition}%[section]
 
\theoremstyle{remark}
\newtheorem*{remark}{Remark}

%\setcounter{section}{u}

\begin{document}

\section{Riemann and Lebesgue integrals}
%\section{Classes of sets}
%\setcounter{subsection}{1}

Recall that the definition for a Riemann integral for a bounded function $f:[a,b]\mapsto\RR$, with a partition $\Delta : a = x_0<x_1<\dots<x_n=b$, define the meshsize, $|\Delta|$ to be
\\

{\centering $|\Delta| = $ max $\{\Delta x_k := x_k - x_{k-1}, k\in 1..n\}$ \par}

\definition{The Riemann integral of $f$ on $[a,b]$ is defined as:
\\

{\centering $(R) \displaystyle\int^b_af(x)dx = \lim_{|\Delta| \to 0} \displaystyle\sum^n_{k=1}f(\xi_k) \Delta x_k$  \par}

where $\xi_k \in [x_{k-1}, x_k], k = 1..n$

}

\end{document}















