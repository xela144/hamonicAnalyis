\documentclass[12pt]{article}

% This first part of the file is called the PREAMBLE. It includes
% customizations and command definitions. The preamble is everything
% between \documentclass and \begin{document}.

\usepackage[margin=1in]{geometry}  % set the margins to 1in on all sides
\usepackage{graphicx}              % to include figures
\usepackage{amsmath}               % great math stuff
\usepackage{amsfonts}   
           % for blackboard bold, etc
\usepackage{amsthm}    % better theorem environments

%\usepackage[demo]{graphicx}
%\usepackage{caption}
%\usepackage{subcaption}

% to anchor figures
%\usepackage{float}    
%\restylefloat{figure}

% various theorems, numbered by section

\newtheorem{thm}{Theorem}[section]
\newtheorem{lem}[thm]{Lemma}
\newtheorem{prop}[thm]{Proposition}
\newtheorem{cor}[thm]{Corollary}
\newtheorem{conj}[thm]{Conjecture}

\DeclareMathOperator{\id}{id}

\newcommand{\bd}[1]{\mathbf{#1}}  % for bolding symbols
\newcommand{\mc}[1]{$\mathcal{#1}$}  % for bolding symbols - no math mode
\newcommand{\MC}[1]{\mathcal{#1}}  % for bolding symbols
\newcommand{\RR}{\mathbb{R}}      % for Real numbers
\newcommand{\ZZ}{\mathbb{Z}}      % for Integers
\newcommand{\NN}{\mathbb{N}}      % for Naturals 
\newcommand{\col}[1]{\left[\begin{matrix} #1 \end{matrix} \right]}
\newcommand{\comb}[2]{\binom{#1^2 + #2^2}{#1+#2}}



%%%% "fancy header"
\usepackage{fancyhdr}   %from donez
\pagestyle{fancy}
\headheight 15pt
\rhead{Fall 2015}
\chead{Ch 2. Measure Theory}
\lhead{Real Analysis with Wavelets}
\rfoot{}
\cfoot{\thepage}
\lfoot{}
\renewcommand{\footrulewidth}{0pt}
%\renewcommand{\headrulewidth}{0pt}
%%%%%

%\usepackage{lmodern}
%\renewcommand{\sfdefault}{lmss}
%\usepackage[T1]{fontenc}
%
%
%\makeatletter
%\special{papersize=\the\paperwidth,\the\paperheight}
%
%\usepackage{lipsum}
%\usepackage{marginnote}
%\usepackage[top=1.5cm, bottom=2cm, outer=5cm, inner=2cm, heightrounded, marginparwidth=4.2cm, marginparsep=.5cm]{geometry}
%
%\makeatother
%\usepackage[backend=biber, style=brent]{biblatex}

%\pagestyle{empty}

%%%% cut here

%\usepackage{paralist}
%\renewenvironment{thebibliography}[1]{%
%  \section*{\refname}%
%  \let\par\relax\let\newblock\relax%
%  \inparaenum[{[}1{]}]}{\endinparaenum}
%\renewcommand{\bibitem}[1]{\item}
%\makeatletter
%\renewenvironment{thebibliography}[1]
%     {\section*{\refname}%
%      \@mkboth{\MakeUppercase\refname}{\MakeUppercase\refname}%
%      \begin{inparaenum}%
%      \list{\@biblabel{\@arabic\c@enumiv}}%
%           {\settowidth\labelwidth{\@biblabel{#1}}%
%           \leftmargin\z@  \labelwidth
%           % \leftmargin\labelwidth
%            \advance\leftmargin\labelsep
%            \@openbib@code
%            \usecounter{enumiv}%
%            \let\p@enumiv\@empty
%            \renewcommand\theenumiv{\@arabic\c@enumiv}}%
%      \sloppy
%      \clubpenalty4000
%      \@clubpenalty \clubpenalty
%      \widowpenalty4000%
%      \sfcode`\.\@m}
%     {\def\@noitemerr
%       {\@latex@warning{Empty `thebibliography' environment}}%
%      \endlist\end{inparaenum}}
%\renewcommand\newblock{}
%\makeatother

%%%% cut here 

\usepackage{enumerate}

\theoremstyle{definition}
\newtheorem{definition}{Definition}%[section]
 
\theoremstyle{remark}
\newtheorem*{remark}{Remark}

%\setcounter{section}{u}

\begin{document}

\section{Classes of sets}
%\section{Classes of sets}
%\setcounter{subsection}{1}
\definition{A nonempty collection $\MC{R}$ of subsets of $X$ is called a \textit{ring} on $X$ if for any $E_1$, $E_2 \in \MC{R}$ we have $E_1\cup E_2 \in \MC{R}$ and $E_1 \setminus E_2 \in \MC{R}$. A ring on $X$ is called a \textit{$\sigma$-ring} if it is closed under countabl unions, that is, if for $E_k \in \MC{R}, k \in \NN$, then $\cup^{\infty}_{k=1}E_k\in\MC{R}$
\begin{enumerate}[(a)]
\item If \mc{R} is a ring, then $\emptyset\in\MC{R}$ 
\item A,B $\in\MC{R} \rightarrow A\setminus B \in \MC{R}$ and $A\Delta B \in \MC{R}$.
\item $E_k \in \MC{R}, k\in \{1..n\}\rightarrow  \bigcup\limits_{k=1}^{n} E_k \subset \MC{R}$, and $ \bigcap\limits_{k=1}^{n} E_k \subset\MC{R}$
\item  $\bigcap\limits_{k=1}^{n} E_k \in \MC{R}$ since $\bigcap\limits_{k=1}^{n} E_k  = E\setminus \bigcup\limits_{k=1}^{n} (E \setminus E_k )$, that is, a $\sigma$-ring is closed under countable $\cap$
\item \mc{R} is a $\sigma$-ring, and $E_k\in\MC{R}$ for $k\in\NN$, then $\varliminf_{n \to \infty} E_k \in \MC{R}$ and  $\varlimsup_{n \to \infty} E_k \in \MC{R}$ 
\item $\MC{C} \subset 2^X \rightarrow \exists \MC{R}(\MC{C}):MC{C}\subset\MC{R}(\MC{C})$ and $\forall \MC{C}\subset\MC{R}: \MC{R}(\MC{C})\subset\MC{R}$ i.e. $ \MC{R}(\MC{C})$ is the smallest ring containing  \mc{R}.
\item $\MC{C} \subset 2^X \rightarrow \exists \sigma_r(\MC{C}): \MC{C}\subset\sigma_r(\MC{C})$ and $\forall\sigma_r \supset\MC{C} \rightarrow\sigma_r(\MC{C})\subset \sigma_r$, i.e. \mc{R}(\mc{C}) is the smallest ring containing \mc{R}.
\end{enumerate}
}


\section{Measures on a Ring}

\definition{Length function on a collection of sets, \mc{I}:

\begin{enumerate}[(a)]
\item $\lambda(\emptyset) = 0$
\item Monotone: $I\subset J \rightarrow \lambda(I) < \lambda(J)$

\item Finite additivity: $I\in\MC{I}$ and $I=\bigcup\limits_{k=1}^nJ_k$ for $J_k$'s disjoint in $\MC{I}$
\\

{\centering
$\rightarrow \lambda(I) = \sum\limits_{k=1}^n\lambda(J_k)$\par
}

\item Countable subadditivity: $I \in \MC{I}: I\subset \bigcup\limits_{k=1}^{\infty}I_k$ for $I_k \in\MC{I}, k\in\NN$
\\

{\centering
$\rightarrow \lambda(I) \leq \sum\limits_{k=1}^{\infty}\lambda(I_k)$\par
}

\item Countable additivity: $I \in \MC{I}: I\subset \bigcup\limits_{k=1}^{\infty}I_k$ for $I_k \in\MC{I}, k\in\NN$, and mutually disjoint $I_k$'s:
\\

{\centering
$\rightarrow \lambda(I) = \sum\limits_{k=1}^{\infty}\lambda(I_k)$\par
}

\item Translation invariance: $\lambda(I) = \lambda(I+y)$ for $y \in\RR$

\end{enumerate}

}


\definition{$\mu:\MC{R}\mapsto\RR$ is a measure on a ring \mc{R} $\subset 2^X$ if:

\begin{enumerate}[(a)]
\item $\mu(\emptyset) = 0$
\item $\forall E \in \MC{R}, \mu(E) \geq 0$
\item For $E_k$'s disjoint in $\MC{R}: \mu(\bigcup\limits_{k=1}^{\infty}E_k) = \bigcup\limits_{k=1}^{\infty}\mu(E_k)$ 
\end{enumerate}
}

\definition{Measure $m$ on the ring $\MC{R_0}$ on $\RR$, with $E=\cup_{i=1}^n(a_i,b_i]$ with $n \in \NN$ and disjoint $(a_i,b_i]$'s:
\\

{\centering $m(E) =  \sum\limits_{i=1}^n(b_i-a_i)$\par}

}
% $\mu\sim$ measure on a ring \mc{R}, later extended to a $\sigma$-ring and larger collections of sets using outer measure.
\definition{Lebesgue measure: Given a subset $E\subset\RR$, with the length of an interval $I=[a,b]$ given by $\lambda(I) = b-a$, Lebesgue outer measure is defined as
\\

{\centering $m(E) =$ inf$\Bigg\{\displaystyle\sum\limits_{k=1}^{\infty}m(E_k) \Bigg | E_k \in \MC{R_0}$ and $E\subset \cup_{k=1}^{\infty} \Bigg\}$ \par
}

}

\definition{
Sigma ring, $S(\MC{R})$ that contains \mc{R}:
\\

{\centering $S(\MC{R}) = \{E \in 2^X | E\subset \cup^{\infty}_{i=1}E_i$ and $E_i\in\MC{R}, i\in\NN$\}\par
}
\

*For the real line $\RR, S(\MC{R}) = \MC{P}(\RR)$

}


\section{Outer measure and Lebesgue Measure}

\definition{Outer measure: For any set $E\in S(\MC{R})$, define \textit{outer measure}, $\mu^*(E)$ as:

{\centering $\mu^*(E) =$ inf$\Bigg\{\displaystyle\sum\limits_{k=1}^{\infty}m(E_k) \Bigg | E \subset  \cup_{k=1}^{\infty}, E_i \in \MC{R} \Bigg\}$ \par
}


}


\thm{The set $E$ is Lebesgue measurable iff for any $A\subset E$ and $B\subset E^C$:
\\

{\centering $m^*(A\cup B) = m^*(A) + m^*(B)$\par}


}

\thm{ Caratheodory condition: Let $E$ and $A$ be any set in $S(\MC{R})$. Then
\\

{\centering $\mu^*(E) = \mu*(E\cap A) + \mu^*(E\setminus A)$

}


}

\end{document}















